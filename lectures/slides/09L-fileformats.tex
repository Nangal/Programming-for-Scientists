\documentclass{beamer}
\usetheme{IMM}

\usepackage{pgf,pgfarrows,pgfnodes,pgfautomata,pgfheaps,pgfshade}
\usepackage{amsmath,amssymb}
\usepackage[utf8]{inputenc}
\usepackage{colortbl}
\usepackage[english]{babel}
\usepackage{booktabs}
\usepackage{slpython}
\usepackage{underscore}
\usepackage{bm}

\author{Luis Pedro Coelho}
\institute{Programming for Scientists}

\graphicspath{{../figures/}{../figures/generated/}{../images/}}

\newcommand*{\code}[1]{\textsl{#1}}
\newcommand*{\Reals}[1]{R}
\newcommand*{\Assign}{\ensuremath{\leftarrow}}
\newcommand{\creditto}[1]{%
\begin{flushright}
(#1)
\end{flushright}%
}

\title{Some Useful File Formats}
\begin{document}
\frame{\maketitle}

\begin{frame}[fragile]
\frametitle{XML: Xtensible Markup Language}
\alert{XML} is a 
\begin{itemize}
\item meta-format
\item on which you define formats.
\end{itemize}
\end{frame}

\begin{frame}[fragile]
\frametitle{XML Says}

You define a \alert{hierarchical structure} of tags,\\
with optional attributes, and text.

\bigskip

A file format is only complete when you specify what the tags mean!
\note{In this way, it's similar to saying you are using a text file for your data.
It only means something if you specify what the text means.}
\end{frame}

\begin{frame}[fragile]
\frametitle{XML Example}
\begin{verbatim}
 <recipe name="bread" prep_time="5m" cook_time="3h">
   <title>Basic bread</title>
   <ingredient amount="8dL">Flour</ingredient>
   <ingredient amount="10g">Yeast</ingredient>
   <ingredient amount="4dl">Water</ingredient>
   <ingredient amount="1tsp">Salt</ingredient>
   <instructions>
     <step>Mix all ingredients together.</step>
     <step>Knead thoroughly.</step>
     <step>Cover with a cloth, and wait 1h.</step>
     <step>Knead again.</step>
     <step>Place in a bread baking tin.</step>
     <step>Bake in the oven at 180C for 30 min.</step>
   </instructions>
 </recipe>
\end{verbatim}
\begin{flushright}
(Wikipedia)
\end{flushright}
\end{frame}

\begin{frame}[fragile]
\frametitle{XML}

\begin{block}{XML Pros \& Cons}
\begin{itemize}
\item Appropriate for \alert{heavy-duty} formats
\item Based on standards
\item Parseable in any programming language
\item extensible, mix-'n'-match
\item XML has a lot attached to it.\\
    It would take us more than one lecture on how to\\
    parse that simple example!
\end{itemize}
\end{block}

\end{frame}

\begin{frame}[fragile]
\frametitle{Comma-Separated Values}
\begin{itemize}
\item Simple idea
\item Messy details
\end{itemize}

Great for communicating with a spread-sheet application.
\end{frame}

\begin{frame}[fragile]
\frametitle{CSV Example}

\begin{verbatim}
luispedro,Luis Pedro Coelho,lpc@cmu.edu
rita,Rita Reis,rita@somewhere.com
\end{verbatim}
\end{frame}

\begin{frame}[fragile]
\frametitle{Messy Issues}
\begin{itemize}
\item Everything we talked about last time comes into play again\\
    (file enconding issues, line-endings,\ldots)
\item What if you want to have a comma inside a field?
\item What if you want to have a newline inside a field?
\item \ldots
\end{itemize}
\end{frame}

\begin{frame}[fragile]
\frametitle{CSV: Comma Separated Files} 

\begin{python}
import csv
for name,user,email in csv.reader(file('emails.csv')):
    print name,user,email
\end{python}

\end{frame}

\begin{frame}[fragile]
\frametitle{Python Pickle}

\begin{python}
import pickle

obj = ...

pickle.dump(obj,file('output.pp','w'))

obj2 = pickle.load(file('output.pp'))
\end{python}
\end{frame}

\begin{frame}[fragile]
\frametitle{Gzip}

\begin{python}
import pickle
from gzip import GzipFile

obj = ...

pickle.dump(obj,GzipFile('output.pp.gz','w'))
obj2 = pickle.load(GzipFile('output.pp'))
\end{python}
\end{frame}

\begin{frame}[fragile]
\frametitle{Numpy Files (npy)}

\begin{python}
import numpy as np

A = np.arange(100).reshape((10,10))
np.save('output.npy',A)

B = np.load('output.npy')
\end{python}
\end{frame}

\begin{frame}[fragile]
\frametitle{JSON: JavaScript Object Notation}

\begin{itemize}
\item Strings
\item Lists
\item String $\to$ Object Dictionaries
\end{itemize}

\begin{python}
import simplejson

object = ['One','Two','Three']
print simplejson.dumps(object)
\end{python}
prints ["One", "Two", "Three"]

\end{frame}

\begin{frame}[fragile]
\frametitle{JSON II}
\begin{python}
import simplejson
complex = (range(3),
            {'Name': 'Luis',
            'Emails' : ['lpc@cmu.edu',
                    'lcoelho@andrew.cmu.edu']})
print simplejson.dumps(complex)
\end{python}
prints 
\begin{python}
[[0, 1, 2], 
    {"Name": "Luis",
    "Emails": ["lpc@cmu.edu", "lcoelho@andrew.cmu.edu"]}]
\end{python}
\end{frame}

\begin{frame}[fragile]
\frametitle{Project Configuration}

\begin{block}{Options}
\begin{itemize}
\item Which model to use for generation? (and parameters)
\item Which model to use for image? (and parameters)
\item Which method for detection? (and parameters)
\item Which method for tracking? (and parameters)
\item Which method for visualisation? (and parameters)
\item Which statistics?
\end{itemize}
\end{block}

\end{frame}

\begin{frame}[fragile]
\frametitle{INI File}
\begin{verbatim}
[Generation]
method=brownian
std=1.2
[Image-Generation]
method=single
shot-noise=2.0
[Detection]
method=otsu
filter-smaller=2
[Tracking]
method=hungarian
[Visualization]
method=colors
[Statistics]
speed=True
velocity=True
\end{verbatim}
\end{frame}

\begin{frame}[fragile]
\frametitle{Desired Characteristics}

\begin{itemize}
\item Read-able
\item Edit-able
\item Parse-able
\item Standard
\end{itemize}

\end{frame}

\begin{frame}[fragile]
\frametitle{Parsing INI Files}

\begin{python}
import ConfigParser
config = ConfigParser.ConfigParser()
config.readfp(file('particles.ini'))
generation_method = config.get('Generation','method')
generation_options = dict(config.items('Generation'))

print generation_options
\end{python}
\end{frame}

\begin{frame}[fragile]
\frametitle{Particles}

\begin{python}
import ConfigParser
import optparse
parser = optparse.OptionParser()
parser.add_option('--config-file',
                action='store',
                dest='configfile',
                default='particles.ini')
options,args = parser.parse_args()
config = ConfigParser.ConfigParser()
config.readfp(file(options.configfile))
\end{python}
\end{frame}

\begin{frame}[fragile]
\frametitle{Summary}

\begin{itemize}
\item For your own stuff: Python pickles
\item (Or, if it's only numpy arrays: npy)
\item For simple communication/editing: INI files or JSON
\item For heavy-duty publication: standard XML-based format
\end{itemize}
\end{frame}

\end{document}
