\documentclass{beamer}
\usetheme{IMM}

\usepackage{pgf,pgfarrows,pgfnodes,pgfautomata,pgfheaps,pgfshade}
\usepackage{amsmath,amssymb}
\usepackage[utf8]{inputenc}
\usepackage{colortbl}
\usepackage[english]{babel}
\usepackage{booktabs}
\usepackage{slpython}
\usepackage{underscore}
\usepackage{bm}

\author{Luis Pedro Coelho}
\institute{Programming for Scientists}

\graphicspath{{../figures/}{../figures/generated/}{../images/}}

\newcommand*{\code}[1]{\textsl{#1}}
\newcommand*{\Reals}[1]{R}
\newcommand*{\Assign}{\ensuremath{\leftarrow}}
\newcommand{\creditto}[1]{%
\begin{flushright}
(#1)
\end{flushright}%
}

\title{File Parsing}
\begin{document}
\frame{\maketitle}

\begin{frame}[fragile]
\frametitle{File Representions}
\begin{block}{What's a File?}
A sequence of bytes.

\bigskip
(and meta-data).
\note{This is very unixy. Files in Windows can have multiple streams}
\end{block}
\end{frame}

\begin{frame}[fragile]
\frametitle{Representing Text}
\begin{block}{ASCII}
\begin{itemize}
\item 65: A
\item 66: B
\item \ldots
\item 48: 0
\item 49: 1
\item \ldots
\item \ldots
\end{itemize}

127 code points taken.
\end{block}
\end{frame}

\begin{frame}[fragile]
\frametitle{File Format Examples: FASTA format}
\begin{verbatim}
> qqq
ACTTTGTTATATATACTATCTGTATTTTC
CTGGGTGAGAGAGTGGTTGAGAGGGGGAA
CCCCCAACCACATTTCCCCACACCCCCTG
ACTTTCCTATATGTCCATTTTTATAATC
> ppp
TTTTTGGTATCTATTTTCCACTCATTCTTTAT
TACCCCAGTCATCACAAAACACACACACAACC
ATTATCTCTAATATATAATTTTACCTTT
\end{verbatim}
\end{frame}

\begin{frame}[fragile]
\frametitle{Parsing FASTA}

\begin{python}
sequences = []
curseq = ''
for line in file('input.fsa'):
    if line[0] == '>':
        sequences.append(curseq)
    else:
        curseq += line.strip()
sequences.append(curseq)
\end{python}
\end{frame}

\begin{frame}[fragile]
\frametitle{File Format Examples (II): GenBank}

\begin{verbatim}
LOCUS       SCU49845     5028 bp    DNA             PLN       21-JUN-1999
DEFINITION  Saccharomyces cerevisiae TCP1-beta gene, partial cds, and Axl2p
            (AXL2) and Rev7p (REV7) genes, complete cds.
ACCESSION   U49845
VERSION     U49845.1  GI:1293613
KEYWORDS    .
SOURCE      Saccharomyces cerevisiae (baker's yeast)
  ORGANISM  Saccharomyces cerevisiae
            Eukaryota; Fungi; Ascomycota; Saccharomycotina; Saccharomycetes;
            Saccharomycetales; Saccharomycetaceae; Saccharomyces.
...
ORIGIN
        1 gatcctccat atacaacggt atctccacct caggtttaga tctcaacaac ggaaccattg
       61 ccgacatgag acagttaggt atcgtcgaga gttacaagct aaaacgagca gtagtcagct
      121 ctgcatctga agccgctgaa gttctactaa gggtggataa catcatccgt gcaagaccaa
      181 gaaccgccaa tagacaacat atgtaacata tttaggatat acctcgaaaa taataaaccg
      241 ccacactgtc attattataa ttagaaacag aacgcaaaaa ttatccacta tataattcaa
      301 agacgcgaaa aaaaaagaac aacgcgtcat agaacttttg gcaattcgcg tcacaaataa
      361 attttggcaa cttatgtttc ctcttcgagc agtactcgag ccctgtctca agaatgtaat
\end{verbatim}
\end{frame}

\begin{frame}[fragile]
\frametitle{Line Endings}
\begin{itemize}
\item Unix: LF (line feed)
\item Windows: CRLF (carriage return, line feed)
\item (Old Mac OS: CR)
\end{itemize}

The extra carriage returns will often show up as \^{}M in Unix.

Some unix text files will show up as a single ultra-long line on Windows.

\end{frame}


\begin{frame}[fragile]
\frametitle{What About International Characters?}

\begin{itemize}
\item Such as \'{a} or \c{c}?
\item Or $\mu$?
\item Or Asian characters?\note{I couldn't get asian characters in Latex, for example}
\item Or ---?

\end{itemize}
It's a mess!
\end{frame}

\begin{frame}[fragile]
\frametitle{International Character Sets}

\begin{itemize}
\item Traditional (latin-1,latin-9,latin-15,\ldots)
\item Unicode (16-bits, or 32-bits)
\item UTF-8
\end{itemize}
\end{frame}

\begin{frame}[fragile]
\frametitle{Unicode}

\begin{block}{Unicode}
Use 16~bits for (almost) all possible possible characters.\\
Use 32~bits for all possible characters.
\end{block}

\begin{block}{Byte Order}
If you have a 2~byte number, which byte do you write first?
\end{block}

\end{frame}

\begin{frame}[fragile]
\frametitle{UTF-8}
Emerging standard (at some levels).

\pause

You still see errors

I saw this in an ATM receipt the other day

confirmação (which is confirmação in UTF-8, but printed out in Latin-15)

\end{frame}


\begin{frame}[fragile]
\begin{center}
\begin{tabular}{lllll}
\toprule
Size & Hex Value & Value & Meaning\\
\midrule
2 & 42 4D  &"BM"  &Magic Number (66, 77)\\
4 & 46 00 00 00 & 70 Bytes & Size of Bitmap\\
2 & 00 00 & Unused & Application Specific\\
2 & 00 00 & Unused & Application Specific\\
4 & 36 00 00 00 & 54 bytes & The offset of data.\\
4 & 28 00 00 00 & 40 bytes & Size of header. \\
4 & 02 00 00 00 & 2 pixels & The width in pixels\\
4 & 02 00 00 00 & 2 pixels & The height in pixels\\
2 & 01 00 & 1 plane & Number of color planes.\\
2 & 18 00 & 24 bits & The bits/pixel.\\
4 & 00 00 00 00 & 0 & No compression used\\
4 & 10 00 00 00 & 16 bytes & The size of the raw BMP data\\
4 & 13 0B 00 00 & 2,835 pixels/m& The horizontal resolution\\
4 & 13 0B 00 00 & 2,835 pixels/m& The vertical resolution\\
4 & 00 00 00 00 & 0 & Number of colors in the palette\\
4 & 00 00 00 00 & 0 & Means all colors are important\\
\bottomrule
\end{tabular}
\end{center}
\end{frame}

\begin{frame}[fragile]
\begin{center}
\begin{tabular}{lllll}
\toprule
Size & Hex Value & Value & Meaning\\
\midrule
3 & 00 00 FF & 0 0 255 & Red, Pixel (0,1)\\
3 & FF FF FF & 255 255 255 &  White, Pixel (1,1)\\
2 & 00 00    & 0 & Padding for 4 bytes/row\\
3 & FF 00 00 & 255 0 0 & Blue, Pixel (0,0)\\
3 & 00 FF 00 & 0 255 0 & Green, Pixel (1,0)\\
2 & 00 00    & 0  & Padding for 4 bytes/row\\
\bottomrule
\end{tabular}
\end{center}


\begin{flushright}
(Wikipedia)
\end{flushright}

\end{frame}

\begin{frame}[fragile]
\frametitle{Bitmap}

\centering
\includegraphics{BMPexample}

\end{frame}

\begin{frame}[fragile]
\frametitle{Text vs.\ Binary}
When possible, prefer text formats.

They are simpler.\note{but they still have a million quirks and we'll spend the rest of the lecture talking about them.}
\end{frame}



\note{break here}

\begin{frame}[fragile]
\frametitle{Text Based Formats}

\begin{itemize}
\item Text format is not only for text
\item Your Python files are text files
\item Anything can be represented as a text-file
\end{itemize}

\end{frame}

\begin{frame}[fragile]
\frametitle{Text Formats}

\begin{itemize}
\item Easier to parse
\item Easier to debug
\item Easier to transfer from one machine to the next
\item Larger files (but you can compress them)
\end{itemize}
\end{frame}

\begin{frame}[fragile]
\frametitle{Tabular Text Formats}

\begin{itemize}
\item Comma- or TAB- separated files are flexible
\item Can be used to transfer from and to spreadsheet software
\item Unfortunately, these are often not 100\% well specified
\end{itemize}
\end{frame}

\begin{frame}[fragile]
\frametitle{Example: SAM Format}
\begin{verbatim}

@HD     VN:1.0  SO:unsorted
@SQ     SN:     LN:9732
@PG     ID:bowtie2      PN:bowtie2      VN:2.0.0-beta7
FC5:1101:1441:2131        4       *       0       0       *       *       0       0       TNTCTATTTCGTGTTCATGCAAATTCTGTTGCTGGTCTGGGTGACTATT       _BP\acccgggbeghfhhfhhf_gggghhfghhfhdfgfffbfgghdeg      YT:Z:UU
FC5:1101:1371:2227        4       *       0       0       *       *       0       0       TTGTCTCTGGTCCTATTTTTCATGGAGTCTTTTTTTTTTTCACTTTTAA       bbbeeeeeggeggihhiiiifgiiiegghhiiiiiiiiiiZZ\cghgd`      YT:Z:UU
FC5:1101:1650:2171        4       *       0       0       *       *       0       0       GTGTAATTCTTATTAATGTTATAGAAAATAGTAAGATAAAATAGTTCAG       ba_eeeeegggggihiiiiifihhiiiiiiieghhhhifdeddfdghhi      YT:Z:UU
FC5:1101:1599:2183        4       *       0       0       *       *       0       0       GTATATACTCAAACAGAAAGAGGAGGGGTGGTAGAGATGAGAATTTTAC       babeeeeegggggiiihiiiegheghii^cgaegfgghfffhhiiiihi      YT:Z:UU
FC5:1101:1539:2201        4       *       0       0       *       *       0       0       CCTTCTCAGAACCACTAAGGCAGCAGTTTGACGTTGACAATTGAAGTTT       _^^cccdeggacgffgghhhZYbcge[bcgfgh\fgegfhfhfcgfI^f      YT:Z:UU
\end{verbatim}
\end{frame}

\begin{frame}[fragile]
\frametitle{SAM Close Up}

\begin{verbatim}
FC5:1101:1441:2131\t4\t*\t0\t0...
\end{verbatim}

\end{frame}

\begin{frame}[fragile]
\frametitle{Parsing This File}

\begin{python}
ifile = open('input.sam')
matched = 0
nreads = 0
for line in ifile:
    if line[0] == '@':
        continue
    nreads += 1
    tokens = line.strip().split('\t')
    if tokens[1] != '4': # Four is *unmatched*
        matched += 1

print 'Matched {0} reads out of {1} ({2:.2%})' \
            .format(matched, nreads, matched/float(nreads))
\end{python}
\end{frame}

\begin{frame}[fragile]
\frametitle{Conclusions}
\begin{itemize}
\item Use UTF-8
\item Prefer text-based formats (use generic compression on top)
\end{itemize}
\end{frame}

\end{document}
