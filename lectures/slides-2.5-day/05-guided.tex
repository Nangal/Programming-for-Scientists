\documentclass{beamer}
\usetheme{IMM}

\usepackage{pgf,pgfarrows,pgfnodes,pgfautomata,pgfheaps,pgfshade}
\usepackage{amsmath,amssymb}
\usepackage[utf8]{inputenc}
\usepackage{colortbl}
\usepackage[english]{babel}
\usepackage{booktabs}
\usepackage{slpython}
\usepackage{underscore}
\usepackage{bm}

\author{Luis Pedro Coelho}
\institute{Programming for Scientists}

\graphicspath{{../figures/}{../figures/generated/}{../images/}}

\newcommand*{\code}[1]{\textsl{#1}}
\newcommand*{\Reals}[1]{R}
\newcommand*{\Assign}{\ensuremath{\leftarrow}}
\newcommand{\creditto}[1]{%
\begin{flushright}
(#1)
\end{flushright}%
}

\title{Guided Exercises}
\begin{document}
\frame{\maketitle}

\begin{frame}
\frametitle{Goals for this hour}

\begin{itemize}
\item A quiz
\item Do a few exercises.
\item Play around.
\item You can work alone, in pairs, in triples,\ldots
\end{itemize}

\end{frame}

\begin{frame}
\frametitle{Lists I}

How do you access the first element of a list?

Assume \texttt{list} is a list:

\begin{enumerate}
\item \texttt{list[1]}
\item \texttt{list[0]}
\item \texttt{list[-1]}
\item \texttt{list(0)}
\item \texttt{list(-1)}
\item \texttt{list(1)}
\end{enumerate}
\end{frame}

\begin{frame}
\frametitle{Lists II}

How do you access the last element of a list?

Assume \texttt{list} is a list:

\begin{enumerate}
\item \texttt{list[1]}
\item \texttt{list(-0)}
\item \texttt{list[-1]}
\item \texttt{list(-1)}
\item \texttt{list(1)}
\item \texttt{list[-0]}
\end{enumerate}
\end{frame}

\begin{frame}
{}

\bigskip
\bigskip
\bigskip
Exercises
\end{frame}

\begin{frame}[fragile]
\frametitle{Object Identity}

What is the difference between the following two code examples:

A)
\begin{python}
A = [1, 2, 3]
B = [1, 2, 3]
\end{python}

B)

\begin{python}
A = [1, 2, 3]
B = A
\end{python}

Write a small piece of code (should be 2 or 3 lines) that behaves differently
if you insert it after each of the two segments above.

\pause

\begin{python}
B[0] = 0
print A
\end{python}

\end{frame}


\begin{frame}[fragile]
\frametitle{sum}
\begin{enumerate}
\item Learn about the built-in function \lstinline{sum}
\item Write an implementations of this function
\end{enumerate}

\pause
\begin{python}
def sum(xs, start=0):
    '''
    s = sum(xs, start=0)

    Returns the sum of all values in ``xs`` + ``start`` (which defaults to 0)
    '''
    for x in xs:
        start += x
    return start
\end{python}
\end{frame}

\begin{frame}[fragile]
\frametitle{Numpy}

\begin{python}
import numpy as np
\end{python}
\end{frame}


\begin{frame}[fragile]
\frametitle{Matplotlib}

\begin{python}
import numpy as np
from matplotlib import pyplot as plt

X = np.linspace(-4, 4, 100)
Y = np.exp(.5-X*X)
\end{python}

\centering
\includegraphics[width=.5\textwidth]{gaussian.pdf}

\end{frame}



\end{document}
