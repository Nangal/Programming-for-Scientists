\documentclass{beamer}
\usetheme{IMM}

\usepackage{pgf,pgfarrows,pgfnodes,pgfautomata,pgfheaps,pgfshade}
\usepackage{amsmath,amssymb}
\usepackage[utf8]{inputenc}
\usepackage{colortbl}
\usepackage[english]{babel}
\usepackage{booktabs}
\usepackage{slpython}
\usepackage{underscore}
\usepackage{bm}

\author{Luis Pedro Coelho}
\institute{Programming for Scientists}

\graphicspath{{../figures/}{../figures/generated/}{../images/}}

\newcommand*{\code}[1]{\textsl{#1}}
\newcommand*{\Reals}[1]{R}
\newcommand*{\Assign}{\ensuremath{\leftarrow}}
\newcommand{\creditto}[1]{%
\begin{flushright}
(#1)
\end{flushright}%
}

\title{Testing}
\usepackage{pgfpages}
\setbeameroption{show notes}
\setbeameroption{show notes on second screen}

\begin{document}
\frame{\maketitle}

\begin{frame}[fragile]
\frametitle{Defensive Programming}

Defensive programming means writing code that will catch bugs early.
\end{frame}

\begin{frame}[fragile]
\frametitle{Remember the Homework?}

\begin{python}
def trim(qs, thresh):
    . . .
    assert len(qs) > 0, 'trim: got empty list'
\end{python}
\end{frame}

\begin{frame}[fragile]
\frametitle{Assert}
\begin{python}
assert <condition>, <error message>
\end{python}

\note{
    Catch errors early

    Give good error messages
}
\end{frame}

\begin{frame}[fragile]
\frametitle{Testing}

Do you test your code?

\note{
    Of course you do, interactively, informally.

    Here we are only going to make this automatic.
}
\end{frame}

\begin{frame}[fragile]
\frametitle{Testing trim}
\begin{python}


import numpy as np
from trimfq import trim

qs = np.array([])
trim(qs, 20)

qs = np.array([20,20])
trim(qs, 20)
\end{python}

\pause
These simple sort of tests are called \alert{smoke tests}.
\note{
    A ``real'' smoke test is when you put smoke through pipes and check for leaks
}
\end{frame}

\begin{frame}[fragile]
\frametitle{Testing trim II}
\begin{python}

qs = np.array([10,10,10,20,20,20,20,10])
s,e = trim(qs, 15)
assert np.all(qs[s:e] >= 15)
assert s < e
\end{python}
\end{frame}

\begin{frame}[fragile]
\frametitle{Testing trim III}

Where are errors likely to lurk?
\pause

\begin{itemize}
\item At the edges?
\item What if the whole string is \alert{above} threshold?
\item What if the whole string is \alert{below} threshold?
\end{itemize}

\end{frame}

\begin{frame}[fragile]
\frametitle{Testing trim IV}
\begin{python}

s,e = trim(np.array([10,10,10,10]), 5)
assert s == 0
assert e == 4

s,e = trim(np.array([10,10,10,10]), 15)
assert s == e # Note that we 
              # DO NOT care about
              # actual values

\end{python}
\end{frame}

\begin{frame}[fragile]
\frametitle{Testing trim V}
\begin{python}

s,e = trim(np.array([10,10,20,20]), 15)
assert s == 2
assert e == 4

s,e = trim(np.array([20,20,10,10]), 15)
assert s == 0
assert e == 2
\end{python}
\end{frame}

\begin{frame}[fragile]
\frametitle{Fencepost Errors}

If you build a straight fence 100 meters long with posts 10 meters apart, how
many posts do you need?

\pause

\bigskip
\bigskip

Eleven, but we often think 10.

\end{frame}

\begin{frame}[fragile]
\frametitle{What is the use of testing?}

\begin{itemize}
\item Ok, I tested it
\item It seems to work
\item Now, I am happy
\pause
\item But save those tests!
\end{itemize}
\end{frame}

\begin{frame}[fragile]
\frametitle{When your code changes}

\begin{itemize}
\item When your code changes\ldots
\pause
\item \ldots you rerun your tests.
\item Over time, you will accumulate a collection of tests.
\end{itemize}
\end{frame}

\begin{frame}[fragile]
\frametitle{Software Testing Philosophies}

\begin{enumerate}
\item Test everything. Test it twice.
\item Write tests first.
\item Regression testing.
\end{enumerate}
\end{frame}

\begin{frame}[fragile]
\frametitle{Regression Testing}

Make sure bugs only appear once!

\end{frame}

\begin{frame}[fragile]
\frametitle{Nose testing}

\begin{itemize}
\item Many utilities already exist to help manage test suites\\
(A test suite is a fancy name for ``a bunch of tests).
\item In Python, \alert{nose} is the most popular one.
\end{itemize}

\url{http://nose.readthedocs.org/en/latest/}

\note{
    Demo it.

    Show Coverage HTML
}

\end{frame}



\end{document}
