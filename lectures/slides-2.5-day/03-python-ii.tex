\documentclass{beamer}
\usetheme{IMM}

\usepackage{pgf,pgfarrows,pgfnodes,pgfautomata,pgfheaps,pgfshade}
\usepackage{amsmath,amssymb}
\usepackage[utf8]{inputenc}
\usepackage{colortbl}
\usepackage[english]{babel}
\usepackage{booktabs}
\usepackage{slpython}
\usepackage{underscore}
\usepackage{bm}

\author{Luis Pedro Coelho}
\institute{Programming for Scientists}

\graphicspath{{../figures/}{../figures/generated/}{../images/}}

\newcommand*{\code}[1]{\textsl{#1}}
\newcommand*{\Reals}[1]{R}
\newcommand*{\Assign}{\ensuremath{\leftarrow}}
\newcommand{\creditto}[1]{%
\begin{flushright}
(#1)
\end{flushright}%
}

\title{Python's Odds and Ends}
\begin{document}
\frame{\maketitle}

\begin{frame}[fragile]
\frametitle{Python So Far}

\begin{block}{Python}
\begin{enumerate}
\item Basic types: int, float, list
\item Control flow: for, while, if, else, elif
\end{enumerate}
\end{block}

\end{frame}

\begin{frame}[fragile]
\frametitle{List Indexing}

\begin{python}
students = ['Luis','Rita','Sabah','Grace']
print students[0]
print students[1:2]
print students[1:]
print students[-1]
print students[-2]
\end{python}
\end{frame}

\begin{frame}[fragile]
\frametitle{Tuples (I)}
\begin{python}
A = (0,1,2)
B = (1,)

print A[0]
print len(B)
\end{python}

\end{frame}

\begin{frame}[fragile]
\frametitle{Tuples (II)}
Tuples are like \alert{immutable} lists.
\end{frame}

\begin{frame}[fragile]
\frametitle{Dictionaries}

\begin{itemize}
\item Dictionaries are \alert{associative arrays}.
\end{itemize}

\begin{python}
gene2ensembl = {}
gene2ensembl['SMAD9'] = 'ENSG00000120693'
gene2ensembl['ZNF670'] = 'ENSG00000135747'

print gene2ensembl['SMAD9']
\end{python}
\end{frame}

\begin{frame}[fragile]
\frametitle{Set Type}
\begin{python}
numbers = set([1,2,5])
print 3 in numbers
numbers.add(4)
print numbers
numbers.add(1)
print numbers
print numbers | set(['Rita'])
print numbers - set([2,3])
\end{python}

Output:
\begin{python}
False
set([1, 2, 4, 5])
set([1, 2, 4, 5])
set([1, 2, 4, 5, 'Rita'])
set([1, 4, 5])
\end{python}

\end{frame}

\begin{frame}[fragile]
\frametitle{Frozenset Type}

\begin{python}
numbers = frozenset([1,2,5])

print 3 in 5 # False
print 2 in 5 # True

numbers.add(1) # ERROR!!
\end{python}

\end{frame}

\begin{frame}[fragile]
\frametitle{What's up With Immutability?}

\pause
You can only use \alert{immutable} objects as dictionary keys!

\end{frame}

\begin{frame}[fragile]
\frametitle{Complex Numbers}
\begin{python}
A = 1+1j
print A**2
print A**4
\end{python}

prints

\begin{python}
2j
(-4+0j)
\end{python}
\end{frame}


\begin{frame}[fragile]
\frametitle{None object}

\begin{python}
None
\end{python}
\end{frame}

\begin{frame}[fragile]
\frametitle{Object Identity}

\begin{block}{Object Identity}
\begin{itemize}
\item \lstinline{A is B}
\item \lstinline{A is not B}
\item \lstinline{id(obj)}
\end{itemize}
\end{block}

\end{frame}

\begin{frame}[fragile]
\frametitle{List Comprehensions}

\begin{python}
name = [ <expr> for <name> in <sequence> if <condition> ]
\end{python}

maps to

\begin{python}
name = []
for <name> in <sequence>:
    if <condition>:
        name.append(<expr>)
\end{python}

\end{frame}

\begin{frame}[fragile]
\frametitle{List Comprehensions Example}

\begin{python}
squares = [x*x for x in xrange(1,20)]
evensquares = [x*x for x in xrange(1,20) if (x%2) == 0]
\end{python}

\begin{python}
squares = []
for x in xrange(1,20):
    squares.append(x*x)

evensquares = []
for x in xrange(1,20):
    if (x%2) == 0:
        evensquares.append(x*x)
\end{python}

\end{frame}

\begin{frame}[fragile]
\frametitle{Functions}

\begin{python}
def max(arg0,*args):
    '''
    M = max(arg0,arg1,...)

    Returns the maximum of its arguments
    '''
    M = arg0
    for val in args:
        if val > M:
            M = val
    return M
\end{python}

\end{frame}


\begin{frame}[fragile]
\frametitle{Multiple Assignment}

\begin{python}
A, B = 1,2
\end{python}

Assign multiple elements at once.
\end{frame}

\begin{frame}[fragile]
\frametitle{Multiple Assignment to Return Multiple Arguments}

\begin{python}
def stats(values):
    '''...'''
    return mean(values), std(values)

...
values = ...
props = stats(values)
mu, std = stats(values)
\end{python}

\end{frame}

\begin{frame}[fragile]

\begin{python}
def greet(name,greeting='Hello'):
    '''
    greet(name,greeting='Hello')

    Greets person by name

    Parameters
    ----------
    name: str
        Name
    greeting: str, optional
        Greeting to use
    '''
    print greeting, name

ret = greet('World')

\end{python}

\end{frame}

\begin{frame}[fragile]
\frametitle{Sequences}

\begin{python}
for value in sequence:
    ...
\end{python}

\begin{block}{Sequences}
\begin{itemize}
\item Lists
\item Tuples
\item Sets \& Frozensets
\item Dictionaries
\item ...
\end{itemize}
\end{block}
\end{frame}


\begin{frame}[fragile]
\frametitle{Generators}

\begin{block}{Generator: ``Function''-like Sequence}

\begin{python}
def xrange(start,stop=None,step=None):
    '''
    xrange([start,]stop[,step]) -> xrange object

    Like range, but instead of a list, returns...
    '''
    if stop is None and step is None:
        stop = start
        start = 0
        step = 1
    elif step is None:
        step = 1

    while start < stop:
        yield start
        start += step
\end{python}
\end{block}

\end{frame}

\begin{frame}[fragile]
\frametitle{Generators}

\begin{itemize}
\item Generators are similar to functions, but generate a sequence.
\item Functions use \lstinline{return}, generators use \lstinline{yield}.
\end{itemize}
\end{frame}

\end{document}
