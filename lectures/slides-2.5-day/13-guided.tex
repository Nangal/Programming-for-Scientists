\documentclass{beamer}
\usetheme{IMM}

\usepackage{pgf,pgfarrows,pgfnodes,pgfautomata,pgfheaps,pgfshade}
\usepackage{amsmath,amssymb}
\usepackage[utf8]{inputenc}
\usepackage{colortbl}
\usepackage[english]{babel}
\usepackage{booktabs}
\usepackage{slpython}
\usepackage{underscore}
\usepackage{bm}

\author{Luis Pedro Coelho}
\institute{Programming for Scientists}

\graphicspath{{../figures/}{../figures/generated/}{../images/}}

\newcommand*{\code}[1]{\textsl{#1}}
\newcommand*{\Reals}[1]{R}
\newcommand*{\Assign}{\ensuremath{\leftarrow}}
\newcommand{\creditto}[1]{%
\begin{flushright}
(#1)
\end{flushright}%
}

\title{Guided Exercises}
\usepackage{amsmath}
\begin{document}
\frame{\maketitle}

\begin{frame}
\frametitle{Goals for this hour}

\begin{itemize}
\item A quiz
\item Do a few exercises.
\item Play around.
\item You can work alone, in pairs, in triples,\ldots
\end{itemize}

\end{frame}


\begin{frame}[fragile]
{}


Consider the following code:

\begin{python}
class Point2(object):
    def __init__(self,x,y):
        self.x = x
        self.y = y

    def dist2(self):
        return self.x**2 + self.y**2 

p = Point2(2,2)
print p.dist2()
p.y = 0
print p.dist2()
print p.x
print p.y
\end{python}

This code prints four numbers. What are they:

\begin{enumerate}[a]
\item 8, 8, 2, 0
\item 8, 8, 2, 2
\item 8, 4, 2, 0
\item 8, 4, 2, 2
\end{enumerate}

\end{frame}

\begin{frame}[fragile]
\frametitle{Hello World}

Write a piece of code that writes the string "Hello World" to a file called "hello".

\bigskip
What happens if you use your code above when a file called \textit{hello} already exists?

\end{frame}

\begin{frame}[fragile]
{}
What does the following code do?

\begin{python}
from random import choice
print choice(range(30))
\end{python}

For this, you might want to read the Python documentation.
\end{frame}

\begin{frame}[fragile]
{}
You will sometimes see the following programming idiom:

\begin{python}
import numpy as np
mystery = np.uint32(-1)
\end{python}

Remember:
\begin{itemize}
\item \lstinline{uint32} is short for \alert{unsigned integer of 32 bits}.
\item Unsigned means that it should be interpreted as a positive number.
\item So, \lstinline{mystery} cannot have the value $-1$!
\end{itemize}

What is the value of mystery? Why would we be interested in this particular
value? (Hint: think of its bit representation).

\end{frame}

\begin{frame}[fragile]
\frametitle{File Parsing}

What is a text file?

\begin{enumerate}
\item A file with an extension TXT\\
    (for example file.txt)
\item A file whose content can be interpreted as printable characters.
\item A Word file.
\item A file with text in a human language (like English).
\end{enumerate}
\end{frame}

\begin{frame}[fragile]
\frametitle{Factorial}
Write a factorial function.

\[
N! = N \cdot (N - 1) \cdot (N - 2) \cdots 1
\]

\pause

\[N! = \begin{cases}
        1 & N = 0 \\
        N\cdot (N-1)! & ow.
     \end{cases}
\]

\pause

\begin{python}
def factorial(n):
    assert n >= 0, \
        'factorial is only for n >= 0'
    if n == 0: return 1
    return n * factorial(n-1)
\end{python}

\end{frame}

\begin{frame}[fragile]
\frametitle{Fibonacci}

Write a Fibonnaci function
\begin{align*}
F_0 &= 1\\
F_1 &= 1\\
F_{n+2} &= F_{n+1} + F_n
\end{align*}

\note{
    We can do both a recursive and an iterative function.

    Show how they are equivalent mathematically, but not in performance.
    Handwave notion of big-Oh: ``iterative takes $n$ steps, more or less;
    recursive takes $F_n$ steps; more or less.''
}%
\end{frame}

\end{document}
