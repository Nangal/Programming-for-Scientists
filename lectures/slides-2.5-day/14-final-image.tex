\documentclass{beamer}
\usetheme{IMM}

\usepackage{pgf,pgfarrows,pgfnodes,pgfautomata,pgfheaps,pgfshade}
\usepackage{amsmath,amssymb}
\usepackage[utf8]{inputenc}
\usepackage{colortbl}
\usepackage[english]{babel}
\usepackage{booktabs}
\usepackage{slpython}
\usepackage{underscore}
\usepackage{bm}

\author{Luis Pedro Coelho}
\institute{Programming for Scientists}

\graphicspath{{../figures/}{../figures/generated/}{../images/}}

\newcommand*{\code}[1]{\textsl{#1}}
\newcommand*{\Reals}[1]{R}
\newcommand*{\Assign}{\ensuremath{\leftarrow}}
\newcommand{\creditto}[1]{%
\begin{flushright}
(#1)
\end{flushright}%
}

\title{Review. Future}
\begin{document}
\frame{\maketitle}

\begin{frame}[fragile]
\frametitle{Review}

\begin{block}{Course Content: Python}
\begin{itemize}
\item Basic types: int, float, list, dict, set
\item Control flow: \lstinline{for}, \lstinline{while}, \lstinline{if}, \lstinline{elif}, \lstinline{else}\ldots
\item Defining types: \lstinline{class}, \lstinline{_ _init_ _},\ldots
\item Errors (Exceptions): \lstinline{try}, \lstinline{except}, \lstinline{raise},\ldots
\item Modules \& Standard Library: \lstinline{import}
\end{itemize}
\end{block}

\end{frame}

\begin{frame}[fragile]
\frametitle{Course Content: Numeric Representations}
\begin{block}{Memory \& Numeric Representations}
\begin{itemize}
\item It's bits all the way down
\item Binary representation of signed \& unsigned integers
\item Floating point numbers
\item When handling a lot of data, think of memory usage
\end{itemize}
\end{block}
\end{frame}

\begin{frame}[fragile]
\frametitle{Course Content: Parsing files}

\begin{block}{Parsing files}
\begin{itemize}
\item Files are just Bytes (sequence of small numbers)
\item It is all in how you interpret them
\item There are standard character assignments for text files
\item ASCII (English only), Latin-15 (used in Portugal), UTF-8 (usable for everything).
\end{itemize}
\end{block}
\end{frame}


\begin{frame}[fragile]
\frametitle{Course Content: Open Source}
\begin{block}{Open Source}
\begin{itemize}
\item Free as in beer, free as in speech (gratis/freedom distinction)
\item Copyleft vs.\ liberal licenses
\item It is not about price
\end{itemize}
\end{block}
\end{frame}

\begin{frame}[fragile]
\frametitle{Course Content: Testing}
\begin{block}{Testing}
\begin{itemize}
\item Testing is good and you should do it
\end{itemize}
\end{block}
\end{frame}

\begin{frame}[fragile]
\frametitle{What Was Not in The Course}
\begin{block}{Missing}
\begin{itemize}
\item Some more advanced programming details
\item Version Control
\item Unix \& Shell \& Interacting with Other Programmes
\item More specific tools
\end{itemize}
\end{block}

\note{Break after this}
\end{frame}


\begin{frame}[fragile]
\frametitle{Case Study}

\bigskip
\bigskip
\bigskip
Next Generation Sequencing

\end{frame}


\begin{frame}[fragile]
\frametitle{HTSeq}

\begin{itemize}
\item Third-party libraries are hit \& miss
\item HTSeq is good
\item \url{http://www-huber.embl.de/users/anders/HTSeq/doc/overview.html}
\end{itemize}

\end{frame}

\begin{frame}[fragile]
\frametitle{Examples of HTSeq Usage}

\begin{python}
maxq = 0
for seq in HTSeq.FastqReader('hw-HeLa.fq.gz', 'solexa'):
    maxq = max(maxq, seq.qual.max())

print 'Maximum quality found was {0}.'.format(maxq)
\end{python}

\end{frame}

\begin{frame}[fragile]
\frametitle{Continuing\ldots}

\begin{python}
import HTSeq
allqs = np.array([seq.qual
            for seq in
                HTSeq.FastqReader('hw-HeLa.fq.gz', 'solexa')])
\end{python}

\end{frame}

\begin{frame}[fragile]
\frametitle{Calling System Programmes}

\begin{itemize}
\item Now, you want to align the sequences.
\item Just use \alert{bowtie2}
\end{itemize}


\end{frame}

\begin{frame}[fragile]
\frametitle{Calling External Programmes}

\begin{python}
from os import system

system('bowtie2 -x hg19reference -U {0} -S {1}' \
                .format(ifilename, ofilename))

\end{python}
\end{frame}

\begin{frame}[fragile]
\frametitle{More HTSeq}
\begin{itemize}
\item Parses GFF/GTF files
\item Basis of \texttt{htseq-count}
\end{itemize}
\end{frame}


\begin{frame}[fragile]
\frametitle{Links for Learning More}

\begin{itemize}
\item How to Think Like a Computer Scientist
    \url{http://interactivepython.org/courselib/static/thinkcspy/index.html}
\item Stack Overflow \\
    for asking questions
\item Python.org \& numpy.scipy.org \\
    for documentation
\end{itemize}

\end{frame}

\begin{frame}[fragile]
\frametitle{Any Other Questions\ldots}
Email me: luis@luispedro.org
\end{frame}



\end{document}
