\documentclass{beamer}
\usetheme{IMM}

\usepackage{pgf,pgfarrows,pgfnodes,pgfautomata,pgfheaps,pgfshade}
\usepackage{amsmath,amssymb}
\usepackage[utf8]{inputenc}
\usepackage{colortbl}
\usepackage[english]{babel}
\usepackage{booktabs}
\usepackage{slpython}
\usepackage{underscore}
\usepackage{bm}

\author{Luis Pedro Coelho}
\institute{Programming for Scientists}

\graphicspath{{../figures/}{../figures/generated/}{../images/}}

\newcommand*{\code}[1]{\textsl{#1}}
\newcommand*{\Reals}[1]{R}
\newcommand*{\Assign}{\ensuremath{\leftarrow}}
\newcommand{\creditto}[1]{%
\begin{flushright}
(#1)
\end{flushright}%
}

\title{Python IV}
\begin{document}
\frame{\maketitle}

\section{Exceptions}

\begin{frame}[fragile]
\frametitle{Exceptions}

\begin{block}{Exceptions}
Report errors for higher up.
\end{block}

\end{frame}

\begin{frame}[fragile]
\frametitle{Call Stack}
\begin{python}
def f(x):
    return log(x)**2

def g(x):
    y = f(x)
    return y+1

def h(x):
    return g(x+1) + g(4*x)

print h(0)
\end{python}
\end{frame}

\begin{frame}[fragile]
\frametitle{Exceptions}

\begin{python}
def log(x):
    if x <= 0.:
        raise ValueError(
            'log: argument must be greater than zero')
    ...
\end{python}
\end{frame}

\begin{frame}[fragile]
\frametitle{Try-Except}

\begin{python}
try:
    h(0)
except:
    print 'Ooops'
\end{python}
\end{frame}

\begin{frame}[fragile]
\frametitle{Try-Except}

\begin{python}
try:
    <line 1>
    <line 2>
    <line 3>
except:
    <line 1>
    <line 2>
\end{python}
\end{frame}

\begin{frame}[fragile]
\frametitle{Exceptions}

\begin{block}{Exceptions}
\begin{itemize}
\item Exceptions are \alert{objects}.
\item Exceptions have \alert{type} and \alert{values}.
\end{itemize}
\end{block}
\end{frame}

\begin{frame}[fragile]
\frametitle{Exception Hierarchy}

(Nothing here, folks, look at the blackboard)

\note{
    Draw a diagram on BB:

               Exception
                   |
             StandardError
              |         |
     EnvironmentError  ArithmeticError
       |        |             |
    OSError   IOError     ZeroDivisionError

    Explain that the arrows mean inheritance.
}
\end{frame}

\begin{frame}[fragile]
\frametitle{Exception Handling}

\begin{block}{Exception Handling: Error Handling}
\begin{python}
try:
    <code 1>
    <code 2>
    <code 3>
except IOError, exc:
    print 'I/O Error', exc
except:
    print 'Unspecified error'
\end{python}
\end{block}
\end{frame}

\begin{frame}[fragile]

\begin{python}
def f(x):
    if x <= 0.:
        raise ValueError(
            'f: argument must be greater than zero')
    return sqrt(x)+2

def g(x):
    y = f(x)
    print (y > 2) 

try:
    g(1)
    g(-1)
except:
    print 'Exception'
\end{python}

This outputs:

\begin{columns}
\column{.25\textwidth}
(a)\par
True\\True
\column{.25\textwidth}
(b)\par
True\\False
\column{.25\textwidth}
(c)\par
False\\Exception
\column{.25\textwidth}
(d)\par
True\\Exception
\end{columns}

\end{frame}

\begin{frame}[fragile]
\frametitle{Standard Library Miscellanea}
Random numbers
\begin{itemize}
\item Truly random numbers
\item Pseudo random numbers
\end{itemize}
\end{frame}

\begin{frame}[fragile]
\frametitle{Pseudo Random Numbers}

\[
x_{i+1} = 48271 x_i \mod (2^{31}-1)
\]

\centering
\includegraphics[width=.7\textwidth]{prnwalk}

\end{frame}

\begin{frame}[fragile]
\frametitle{Pseudo Random Numbers}
\begin{itemize}
\item Are not random
\item Some are ``more random'' than others
\end{itemize}
\pause

\begin{itemize}
\item For testing/reproducibility, you want \alert{pseudo-}random numbers.
\item For cryptography, you want really random numbers.
\end{itemize}
\end{frame}

\begin{frame}[fragile]
\frametitle{Testing with random numbers}

\begin{python}
import random
random.seed(32)
for i in xrange(16):
    qs = np.array([random.randint(0,40) for j in xrange(100)])
    s,e = trim(qs, 20)
    assert s <= e
    assert np.all(qs[s:e] > 20)
\end{python}
\end{frame}

\begin{frame}[fragile]
\frametitle{Pickle}

\begin{python}
import pickle

something = [12, 'hello']

pickle.dump(something, open('myfile.pkl', 'w'))
\end{python}

Later

\begin{python}
import pickle

other = pickle.load(open('myfile.pkl'))
\end{python}
\end{frame}
\end{document}
