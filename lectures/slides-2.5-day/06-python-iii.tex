\documentclass{beamer}
\usetheme{IMM}

\usepackage{pgf,pgfarrows,pgfnodes,pgfautomata,pgfheaps,pgfshade}
\usepackage{amsmath,amssymb}
\usepackage[utf8]{inputenc}
\usepackage{colortbl}
\usepackage[english]{babel}
\usepackage{booktabs}
\usepackage{slpython}
\usepackage{underscore}
\usepackage{bm}

\author{Luis Pedro Coelho}
\institute{Programming for Scientists}

\graphicspath{{../figures/}{../figures/generated/}{../images/}}

\newcommand*{\code}[1]{\textsl{#1}}
\newcommand*{\Reals}[1]{R}
\newcommand*{\Assign}{\ensuremath{\leftarrow}}
\newcommand{\creditto}[1]{%
\begin{flushright}
(#1)
\end{flushright}%
}

\title[Python III]{Defining Your Own Types}
\begin{document}

\frame{\maketitle}


\begin{frame}[fragile]
\frametitle{Modules}
We have already seen this
\begin{python}
import module
\end{python}

what is happening exactly?
\end{frame}

\begin{frame}[fragile]
\frametitle{Modules}

\begin{block}{module.py}
\begin{python}
def hello():
    print 'Hello'
\end{python}
\end{block}

\begin{block}{main.py}
\begin{python}
import module
module.hello()
\end{python}
\end{block}

\end{frame}

\begin{frame}[fragile]
\frametitle{Modules}

\begin{block}{module.py}
\begin{python}
def hello():
    print 'Hello'
\end{python}
\end{block}

\begin{block}{main.py}
\begin{python}
import module as mod
mod.hello()
\end{python}
\end{block}

\end{frame}

\begin{frame}[fragile]
\frametitle{Modules}

\begin{block}{module.py}
\begin{python}
def hello():
    print 'Hello'
\end{python}
\end{block}

\begin{block}{main.py}
\begin{python}
from module import hello
hello()
\end{python}
\end{block}

\end{frame}

\begin{frame}[fragile]
\frametitle{Standard Library}
\begin{python}
import datetime
print datetime.datetime.now()
\end{python}
\end{frame}

\begin{frame}[fragile]
\frametitle{Non-Standard Library}
\begin{python}
import numpy
\end{python}
\end{frame}

\begin{frame}[fragile]
\frametitle{User-Defined Types}

\begin{block}{Built-in Types}
\begin{enumerate}
\item lists
\item dictionaries
\item strings
\item \ldots
\end{enumerate}
\end{block}
\end{frame}

\begin{frame}[fragile]
\frametitle{Type}
\begin{block}{What's a Type}
\begin{enumerate}
\item A domain of values
\item A set of methods (functions)
\end{enumerate}
\end{block}

\end{frame}

\begin{frame}[fragile]
\frametitle{Examples of Types}

\begin{block}{List}
\begin{enumerate}
\item Domain: lists
\item Functions: \code{L.append(e),L.insert(idx,e), \ldots}
\item Operators: \code{L[0], 'Rita' in L}
\end{enumerate}
\end{block}

\pause
\begin{block}{Integer}
\begin{enumerate}
\item Domain: $\dots,-2, 1, 0, 1, 2, \dots$
\item Operators: \code{A + B},\ldots
\end{enumerate}
\end{block}
\end{frame}

\begin{frame}[fragile]
\frametitle{User-defined Types}

Object-oriented programming languages allow us to define new types.

\end{frame}

\begin{frame}[fragile]
\frametitle{FastQ Example}

\begin{itemize}
\item DNA (RNA) sequence
\item Quality (integer value) for each position
\end{itemize}

\end{frame}


\begin{frame}[fragile]
\frametitle{FastQ Example}

\begin{python}
def mean(xs):
    return sum(xs)/float(len(xs))
class FastQSequence(object):
    def __init__(self, seq, quals):
        if len(seq) != len(quals):
            print 'OOOOOOOOOOPS!'
        self.seq = seq
        self.quals = quals

    def averageq(self):
        return mean(self.quals)
\end{python}

\end{frame}

\begin{frame}[fragile]
\frametitle{FastQ Example}

\begin{python}
class NAME(object):
    def __init__(self, ...):
        ...

    def METHODNAME(self, . . .):
        . . .
\end{python}

Note: it is a \alert{double underscore}!

\end{frame}

\begin{frame}[fragile]
\frametitle{FastQ Example}

\begin{python}
def mean(xs):
    return sum(xs)/float(len(xs))
class FastQSequence(object):
    def __init__(self, seq, quals):
        if len(seq) != len(quals):
            print 'OOOOOOOOOOPS!'
        self.seq = seq
        self.quals = quals

    def averageq(self):
        return mean(self.quals)

s = FastQSequence('ATTA', [23, 32, 20, 21])
print s.averageq()
\end{python}

\end{frame}

\begin{frame}[fragile]
\frametitle{Exercise}
Take the previous class and a \alert{method} called \lstinline{minq} which
returns the minimum quality.

\end{frame}


\end{document}
